\chapter{Konzeption der Lösung}
\label{kap4}



TODO START

Gut, dass dieses Kapitel schon vorhanden ist.  Sehr hilfreich wäre ein Architekturbild oder eins, das das Zusammenwirken von ROS+Gazebo zeigt, ggf. auch ROS+Roboter und wie sich der Laufplaner in ROS integriert. 
Das gibt den Überblick und dann kann es in die Details der einzelnen Komponenten gehen.
Mir ist klar, dass die erste Realisierung immer von unten nach oben erfolgt. Erst mit Erfahrung lässt sich ein Design von oben nach unten realisieren. Genau das ist für den Leser wichtig: Von der globalen Sicht hinunter zu den Details.
Ich denke, das wird die Hauptbaustelle. Bei Implementierung und Tests sehe ich weniger Probleme. Das ist das, was Sie mit Ihrer Ausbildung an der HS gut beherrschen. Gerade für Ihre Firma ist die konzeptionelle Ebene sehr wichtig. Darüber vermitteln Sie Ideen dem Kunden. Das ist Ihre "Kunst". Bezüglich Umsetzung wird er Ihnen vertrauen. Das ist "Handwerk".
Ich bin überzeugt, dass Sie die Fähigkeit haben, an diesem Punkt richtig gut zu sein.

Zum Architekturbild:
Vielleicht ist hilfreich, auch ein Architekturbild zur alten Lösung zu zeigen, damit der Unterschied deutlich wird.

TODO END

Dieses Kapitel nutzt die in \autoref{kap3} dargestellten verwandten Arbeiten, um ein Konzept für die Portierung des Laufplaners für den Akrobat in \ac{ROS} und Gazebo formal darzustellen. Als Basis dafür werden die Arbeiten von André Herms \autocite{herms2004} und Uli Ruffler \autocite{ruffler2006} herangezogen. Für die Darstellung einer Lösung sind zwei Aspekte von Bedeutung. Dies ist zum einen die Simulationsumgebung. Zum anderen ist dies der Laufplanungsalgorithmus. Beide Aspekte können getrennt voneinander betrachtet werden.

\section{Simulationsumgebung} 

Damit die Simulationsumgebung möglichst nahe an der Software des Laufroboters Akrobat sein soll, soll in beiden Fällen das \ac{ROS} genutzt werden. Um die reale Umgebung zu simulieren, benötigt die Lösung eine Physik-Engine, die bei André Herms \autocite{herms2004} Simulationsumgebung nur rudimentär vorhanden ist. Bei der Nutzung von Gazebo wird standardmäßig die \acf{ODE} mitgeliefert, welche auch hierbei zum Einsatz kommen soll.

Zusätzlich soll es möglich sein, verschiedene Umgebungen laden zu können, damit die Flexibilität beim Testen der Laufalgorithmen erhalten bleibt. Dies ist mit Hilfe von verschiedenen Gazebo-Welten zu lösen. Im ersten Schritt soll dazu eine leere Welt genutzt werden.

Um außerdem ein möglichst reales Szenario zu simulieren, muss ein präzises Robotermodell im Format \ac{URDF} für das \ac{ROS} und Gazebo zum Einsatz kommen, welches das Aussehen, das Kollissionsmodell sowie die Massenträgheit genau beschreibt. Der bestehende Laufplaner besitzt aktuell noch das Modell des \emph{Lauron III}, welches in der neuen Lösung keine Anwendung mehr findet.

Des Weiteren sollen die Gelenkbewegungen nicht mehr über OpenInventor \autocite{inventor} ausgeführt werden. Dazu soll ein \ac{ROS}-spezifisches Paket mit dem Namen \emph{ros\_control} verwendet werden. Durch dieses werden Motoren an den Gelenken simuliert, welche über \ac{ROS}-Topics angesteuert werden können.

\section{Laufplanung}

Da der Laufplaner und die Simulation voneinander getrennt sein sollen, damit das System flexibel für den Austausch von Komponenten bleibt, benötigt der Laufplaner ebenso wie bei André Herms und Uli Ruffler eine xml-Schnittstelle, welche Bewegungen repräsentiert. Die hier dargestellte Schnittstelle soll eine Abwandlung der vorherigen Schnittstellen sein, da diese nicht die Dauer von Bewegungen speichert, sondern lediglich für jeden Schritt die Fußpositionen relativ zur definierten Ausgangsposition. Wie im Vorgängermodell sollen Bewegungen generiert und auch eingelesen werden können.

Der vorherige Laufplaner hat kinematische Berechnungen manuell in OpenInventor durchgeführt. Das neue System soll das \ac{ROS}-Framework \textsc{tf} verwenden, um möglichst einfach Koordinatentransformationen zu berechnen. Die inverse Kinematik wird weiterhin über die analytische Methode berechnet. Allerdings werden dabei die Objektfunktionen von \textsc{tf} verwendet. Weitere geometrische Berechnungen werden ebenfalls über das \textsc{tf}-Framework berechnet.

Des Weiteren muss der Algorithmus auf den Laufroboter Akrobat angepasst werden, da der vorherige Algorithmus auf dem \emph{Lauron III} basiert und die Maße andere sind.