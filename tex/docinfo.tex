% -------------------------------------------------------
% Daten für die Arbeit
% Wenn hier alles korrekt eingetragen wurde, wird das Titelblatt
% automatisch generiert. D.h. die Datei titelblatt.tex muss nicht mehr
% angepasst werden.

% Titel der Arbeit auf Deutsch
\newcommand{\hsmatitelde}{Portierung eines Laufplaners für sechsbeinige Laufroboter in das Robot Operating System}

% Titel der Arbeit auf Englisch
\newcommand{\hsmatitelen}{Transferring of a walk planner for a six-legged walking robot into the robot operating system}

% Weitere Informationen zur Arbeit
\newcommand{\hsmaort}{Mannheim}    % Ort
\newcommand{\hsmaautorvname}{Daniel} % Vorname(n)
\newcommand{\hsmaautornname}{Koch} % Nachname(n)
\newcommand{\hsmadatum}{22.07.2020} % Datum der Abgabe
\newcommand{\hsmajahr}{2020} % Jahr der Abgabe
\newcommand{\hsmafirma}{} % Firma bei der die Arbeit durchgeführt wurde
\newcommand{\hsmabetreuer}{Prof. Dr. Thomas Ihme, Hochschule Mannheim} % Betreuer an der Hochschule
\newcommand{\hsmazweitkorrektor}{} % Betreuer im Unternehmen oder Zweitkorrektor
\newcommand{\hsmafakultaet}{I} % I für Informatik oder E, S, B, D, M, N, W, V
\newcommand{\hsmastudiengang}{IB} % IB IMB UIB IM MTB (weitere siehe titleblatt.tex)

% Zustimmung zur Veröffentlichung
\setboolean{hsmapublizieren}{true}   % Einer Veröffentlichung wird zugestimmt
\setboolean{hsmasperrvermerk}{false} % Die Arbeit hat keinen Sperrvermerk

% -------------------------------------------------------
% Abstract

% Kurze (maximal halbseitige) Beschreibung, worum es in der Arbeit geht auf Deutsch
\newcommand{\hsmaabstractde}{TEST.}

% Kurze (maximal halbseitige) Beschreibung, worum es in der Arbeit geht auf Englisch
\newcommand{\hsmaabstracten}{TEST.}
