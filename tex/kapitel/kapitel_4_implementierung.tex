\chapter{Portierung des Laufplaners nach ROS und Gazebo}

\section{Analyse bestehender Laufplaner}

Herms, Ruffler, und AKrobat GitHub (gibt es dazu eine Arbeit?)

\section{Allgemeiner Aufbau des Pakets}

Die zuvor genannten Pakete müssen nun in das \ac{ROS} portiert werden. Ferner muss der Algorithmus zur Laufplanung mittels Random Sampling extrahiert und mit \ac{ROS} kompatibel gemacht werden. Dabei gibt es einige Herausforderungen, die gelöst werden müssen:



Ordnerstruktur oder vllt in Baustein, Laufzeit und Verteilungssicht?

\section{Aufsetzen der Simulation}

\subsection{Aufsetzen des Roboter-Modells mittels urdf}

Notes: urdf, xacro, Collissions, Inertia + Berechnung + STL-Dateien ( Vereinfachung durch einfaches Geometry Object wenn möglich, sonst vereinfachtes STL, sonst das Original STL) / MeshLab

\subsection{Definition der Gelenkmotoren mittels ros\_control}

Notes: URDF-File, config file, Controller

\subsection{Aufsetzen der Umgebung mittels Gazebo}

launch-files

\subsection{Aufsetzen der Fußsteuerung des Laufroboters}

\section{Aufsetzen von Laufalgorithmen}

\subsection{Implementierung}
\subsection{Generierung von Bewegungen als xml-Datei}
\subsection{Einlesen und Abspielen der Bewegungen}