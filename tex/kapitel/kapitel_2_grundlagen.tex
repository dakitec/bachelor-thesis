\chapter{Grundlagen}
\label{kap2}

\section{Aufbau des Laufroboters}

\section{Robotik}

\subsection{Koordinatensysteme}
\subsection{Direkte Kinematik}

- Mit gegebener Stellung der Gelenke Position und Orientierung des Endeffektes zu berechnen
- Transformationsmatrix, die dann aufgelöst wird und berechnet werden kann

fellmann zitieren

\subsection{Inverse Kinematik}

- Durch Position und Orientierung des Endeffektes mögliche Stellung der Gelenke zu berechnen (meist mehrere Lösungen)
- analytische Berechnung
- numerische Berechnung: lineare Annäherung durch Jacobi-Matrix

fellmann zitieren

\subsection{Laufplanung}

statische Laufalgorithmen, reaktive, planende Laufalgorithmen like RandomSampling

\section{Frameworks}

\subsection{Robot Operating System}
\subsection{Gazebo}
\subsection{MeshLab}